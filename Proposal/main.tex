%%%%%%%%%%%%%%%%%%%%%%%%%%%%%%%%%%%%%%%%%
% Lachaise Assignment
% LaTeX Template
% Version 1.0 (26/6/2018)
%
% This template originates from:
% http://www.LaTeXTemplates.com
%
% Authors:
% Marion Lachaise & François Févotte
% Vel (vel@LaTeXTemplates.com)
%
% License:
% CC BY-NC-SA 3.0 (http://creativecommons.org/licenses/by-nc-sa/3.0/)
% 
%%%%%%%%%%%%%%%%%%%%%%%%%%%%%%%%%%%%%%%%%

%----------------------------------------------------------------------------------------
%	PACKAGES AND OTHER DOCUMENT CONFIGURATIONS
%----------------------------------------------------------------------------------------

\documentclass{article}

%%%%%%%%%%%%%%%%%%%%%%%%%%%%%%%%%%%%%%%%%
% Lachaise Assignment
% Structure Specification File
% Version 1.0 (26/6/2018)
%
% This template originates from:
% http://www.LaTeXTemplates.com
%
% Authors:
% Marion Lachaise & François Févotte
% Vel (vel@LaTeXTemplates.com)
%
% License:
% CC BY-NC-SA 3.0 (http://creativecommons.org/licenses/by-nc-sa/3.0/)
% 
%%%%%%%%%%%%%%%%%%%%%%%%%%%%%%%%%%%%%%%%%

%----------------------------------------------------------------------------------------
%	PACKAGES AND OTHER DOCUMENT CONFIGURATIONS
%----------------------------------------------------------------------------------------

\usepackage{amsmath,amsfonts,stmaryrd,amssymb} % Math packages

\usepackage{enumerate} % Custom item numbers for enumerations

\usepackage[ruled]{algorithm2e} % Algorithms

\usepackage[framemethod=tikz]{mdframed} % Allows defining custom boxed/framed environments

\usepackage{listings} % File listings, with syntax highlighting
\lstset{
	basicstyle=\ttfamily, % Typeset listings in monospace font
}

%----------------------------------------------------------------------------------------
%	DOCUMENT MARGINS
%----------------------------------------------------------------------------------------

\usepackage{geometry} % Required for adjusting page dimensions and margins

\geometry{
	paper=a4paper, % Paper size, change to letterpaper for US letter size
	top=2.5cm, % Top margin
	bottom=3cm, % Bottom margin
	left=2.5cm, % Left margin
	right=2.5cm, % Right margin
	headheight=14pt, % Header height
	footskip=1.5cm, % Space from the bottom margin to the baseline of the footer
	headsep=1.2cm, % Space from the top margin to the baseline of the header
	%showframe, % Uncomment to show how the type block is set on the page
}

%----------------------------------------------------------------------------------------
%	FONTS
%----------------------------------------------------------------------------------------

\usepackage[utf8]{inputenc} % Required for inputting international characters
\usepackage[T1]{fontenc} % Output font encoding for international characters

\usepackage{XCharter} % Use the XCharter fonts

%----------------------------------------------------------------------------------------
%	COMMAND LINE ENVIRONMENT
%----------------------------------------------------------------------------------------

% Usage:
% \begin{commandline}
%	\begin{verbatim}
%		$ ls
%		
%		Applications	Desktop	...
%	\end{verbatim}
% \end{commandline}

\mdfdefinestyle{commandline}{
	leftmargin=10pt,
	rightmargin=10pt,
	innerleftmargin=15pt,
	middlelinecolor=black!50!white,
	middlelinewidth=2pt,
	frametitlerule=false,
	backgroundcolor=black!5!white,
	frametitle={Command Line},
	frametitlefont={\normalfont\sffamily\color{white}\hspace{-1em}},
	frametitlebackgroundcolor=black!50!white,
	nobreak,
}

% Define a custom environment for command-line snapshots
\newenvironment{commandline}{
	\medskip
	\begin{mdframed}[style=commandline]
}{
	\end{mdframed}
	\medskip
}

%----------------------------------------------------------------------------------------
%	FILE CONTENTS ENVIRONMENT
%----------------------------------------------------------------------------------------

% Usage:
% \begin{file}[optional filename, defaults to "File"]
%	File contents, for example, with a listings environment
% \end{file}

\mdfdefinestyle{file}{
	innertopmargin=1.6\baselineskip,
	innerbottommargin=0.8\baselineskip,
	topline=false, bottomline=false,
	leftline=false, rightline=false,
	leftmargin=2cm,
	rightmargin=2cm,
	singleextra={%
		\draw[fill=black!10!white](P)++(0,-1.2em)rectangle(P-|O);
		\node[anchor=north west]
		at(P-|O){\ttfamily\mdfilename};
		%
		\def\l{3em}
		\draw(O-|P)++(-\l,0)--++(\l,\l)--(P)--(P-|O)--(O)--cycle;
		\draw(O-|P)++(-\l,0)--++(0,\l)--++(\l,0);
	},
	nobreak,
}

% Define a custom environment for file contents
\newenvironment{file}[1][File]{ % Set the default filename to "File"
	\medskip
	\newcommand{\mdfilename}{#1}
	\begin{mdframed}[style=file]
}{
	\end{mdframed}
	\medskip
}

%----------------------------------------------------------------------------------------
%	NUMBERED QUESTIONS ENVIRONMENT
%----------------------------------------------------------------------------------------

% Usage:
% \begin{question}[optional title]
%	Question contents
% \end{question}

\mdfdefinestyle{question}{
	innertopmargin=1.2\baselineskip,
	innerbottommargin=0.8\baselineskip,
	roundcorner=5pt,
	nobreak,
	singleextra={%
		\draw(P-|O)node[xshift=1em,anchor=west,fill=white,draw,rounded corners=5pt]{%
		Question \theQuestion\questionTitle};
	},
}

\newcounter{Question} % Stores the current question number that gets iterated with each new question

% Define a custom environment for numbered questions
\newenvironment{question}[1][\unskip]{
	\bigskip
	\stepcounter{Question}
	\newcommand{\questionTitle}{~#1}
	\begin{mdframed}[style=question]
}{
	\end{mdframed}
	\medskip
}

%----------------------------------------------------------------------------------------
%	WARNING TEXT ENVIRONMENT
%----------------------------------------------------------------------------------------

% Usage:
% \begin{warn}[optional title, defaults to "Warning:"]
%	Contents
% \end{warn}

\mdfdefinestyle{warning}{
	topline=false, bottomline=false,
	leftline=false, rightline=false,
	nobreak,
	singleextra={%
		\draw(P-|O)++(-0.5em,0)node(tmp1){};
		\draw(P-|O)++(0.5em,0)node(tmp2){};
		\fill[black,rotate around={45:(P-|O)}](tmp1)rectangle(tmp2);
		\node at(P-|O){\color{white}\scriptsize\bf !};
		\draw[very thick](P-|O)++(0,-1em)--(O);%--(O-|P);
	}
}

% Define a custom environment for warning text
\newenvironment{warn}[1][Warning:]{ % Set the default warning to "Warning:"
	\medskip
	\begin{mdframed}[style=warning]
		\noindent{\textbf{#1}}
}{
	\end{mdframed}
}

%----------------------------------------------------------------------------------------
%	INFORMATION ENVIRONMENT
%----------------------------------------------------------------------------------------

% Usage:
% \begin{info}[optional title, defaults to "Info:"]
% 	contents
% 	\end{info}

\mdfdefinestyle{info}{%
	topline=false, bottomline=false,
	leftline=false, rightline=false,
	nobreak,
	singleextra={%
		\fill[black](P-|O)circle[radius=0.4em];
		\node at(P-|O){\color{white}\scriptsize\bf i};
		\draw[very thick](P-|O)++(0,-0.8em)--(O);%--(O-|P);
	}
}

% Define a custom environment for information
\newenvironment{info}[1][Info:]{ % Set the default title to "Info:"
	\medskip
	\begin{mdframed}[style=info]
		\noindent{\textbf{#1}}
}{
	\end{mdframed}
}
 % Include the file specifying the document structure and custom commands
\usepackage{url}
%----------------------------------------------------------------------------------------
%	ASSIGNMENT INFORMATION
%----------------------------------------------------------------------------------------

\title{Data Visualization: Project Proposal} % Title of the assignment

\author{Feiyu Xiao \quad 2018210441\\ \texttt{xiaofy18@mails.tsinghua.edu.cn}} % Author name and email address

\date{Tsinghua University} % University, school and/or department name(s) and a date

%----------------------------------------------------------------------------------------

\begin{document}

\maketitle % Print the title

%----------------------------------------------------------------------------------------
%	INTRODUCTION
%----------------------------------------------------------------------------------------

\section{Project Content} % Unnumbered section
We introduce a project on \textbf{the exploration of the economic, cultural, scientific landscape of Chengdu.}\\
Chengdu, located in the southwest of China, has been recently acknowledged as one of the fastest growing cities in China. In recent years, its reputation has changed from "Heavenly Land of Plenty " with hot pots and giant pandas to one of the most important engines of economy and technology in China. \textbf{While it is well known the economy and many other things have changed over time, the answer to exactly how those issues have changed is far from clear}. Our main idea is to attempt to answer the question of how the Chengdu's economic, cultural, scientific landscope have changed these years by examining several aspects and utilizing appropriate visualization methods.

\begin{info} % Information block
	Fang Xiaonan and I formed a group to make this proposal and complete the project. And we hope that we will complete the project we have proposed.
\end{info}

%----------------------------------------------------------------------------------------
%	PROBLEM 1
%----------------------------------------------------------------------------------------

\section{Design scheme} % Numbered section
A basic principle of our design is to focus on changes in time and difference in space.
\subsection{Change in times}
According to \textit{Forbes},Chengdu, Chongqing, Suzhou and Nanjing are on the list of the fastest-growing cities in the next 10 years, with Chengdu winning the list with rapid development and developed transportation system. So it is of vital importance to analyze Chengdu's economic data.\\
\\
\textbf{Businesses} are widely considered as a driving force behind Chengdu's rapid growth. So we will then move on to the business data after the overall economic indexes.\\
Apart from the bussinesses, the technology and education are also important. We will continue to observe and study how the technology adn education develop in this region.\\
As is mentioned above, Chengdu has a reputation of its developed and convenient transportation, we will take a short look into the transportation, also including the taxis and the shared bikes.\\
\subsection{Difference in sapce}
However, we cannot ignore \textbf{the huge gap in development between Chengdu and other cities in the Sichuan province}. And we will show that uneven development between regions may hinder the regional long-term development and that is exactly the urgent problem Chengdu is confronted with. \\
\\
In particular, we will conduct the comparison of house prices in major cities in the region and across the country, because house prices and their rising trend are the vane of Economic level and development potential.
\\
\subsection{Bring it all together}
 It's important to consider influences from various sides in order to get a more complete picture. Only by deeply understanding this relationship can we further plan and design the benign development of the city.\\
 \\
 Take the housing market for example, What are the most popular types of businesses in each area? How do the types of businesses in an area correspond with the home values there? What does ultimately this mean for residents who live there? These are all important questions to think about as new businesses move into or grow in Chengdu and as the housing market continues to grow.


%------------------------------------------------
\section{Data Sources and Data Processing}
\subsection{Data Sources}
\subsubsection*{Data from government}
\textbf{National Bureau of Statistics} has made its database public. We can get access to the GDP data, the Industrial index and housing price data of Chengdu, and also we can get the data of other cities for comparision.
\\
\\
Here is the National database:\\
\url{http://data.stats.gov.cn/index.htm}
\\
\\
Chengdu government provides datasets of several aspects, such as transportation data, housing data and employment data. And we can get access to almost all aspects of this city through the database.
\\
\\
Here is the Chengdu public data open platform.\\
\url{http://www.cddata.gov.cn/odweb/index.htm}.\\
Both database provides various formats for downloading, including \textmd{xml,csv}.
\subsubsection*{Data from other platforms}
\textbf{Housing data}: Lianjia is a website with rich and real-time real estate data, and we can easily get historical prices or compare prices in different regions via its API interface of the website\url{https://cd.lianjia.com/}.
\\
\\
\textbf{Employment}: Lagou is a platform for job seekers, so we can get the supply and demand data in the job market which can help us to analyze the industry development and economic expectations. The Lagou website is:\\
\url{https://www.lagou.com/gongsi/252-0-0-0}
\\
\\
\textbf{Diet and entertainment}: Dianping has a large number of user and produce large amounts of data on diet and entertainment. And we can get the evolution of the pattern of diet and entertainment througn analyzing the data which can be get from the website\url{http://www.dianping.com/chengdu/}.

\section{Details}
\subsection{About Us}
The team has two members for now, \textbf{Feiyu Xiao}\footnote{2018210441 \quad  \texttt{xiaofy18@mails.tsinghua.edu.cn}} and \textbf{Fang Xiaonan}\footnote{2018310787 \quad  \texttt{fangxn18@mails.tsinghua.edu.cn}}.\\

This project proposal is for the course \textbf{Data Visualization}, at the Tsinghua University, taught by Professor Zhang Songhai.
\subsection{Amount of work}
Our project is divided into the following three parts: \textbf{Data collection and preprocessing}, \textbf{Coding and Exploration},\textbf{Data interpretation and analysis} and \textbf{
Post treatment}.\\
\begin{itemize}
	\item In \textbf{Data collection and preprocessing}, we collect the data and extract useful information and then do further processing. For example, we will select features which are most representative and explanatory. Also we will conduct data reorganization for further analysis and visualization.
	\item In \textbf{Coding and Exploration}, we conduct the visualization of data.
	\item In \textbf{Data interpretation and analysis}, we organize our results to answer the question \textbf{"how those issues have changed?"}.
	\item In \textbf{Post treatment}, we will build an interactive Website to tell the story and state the answers of the question.
\end{itemize}

\end{document}
